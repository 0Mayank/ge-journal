% The very first letter is a 2 line initial drop letter followed
% by the rest of the first word in caps.
% 
% form to use if the first word consists of a single letter:
% \IEEEPARstart{A}{demo} file is ....
% 
% form to use if you need the single drop letter followed by
% normal text (unknown if ever used by the IEEE):
% \IEEEPARstart{A}{}demo file is ....
% 
% Some journals put the first two words in caps:
% \IEEEPARstart{T}{his demo} file is ....
\IEEEPARstart{I}{n recent} years, gesture-based control systems have emerged as a promising solution for intuitive and contactless interaction with electronic devices. These systems offer particular benefits for individuals with mobility challenges and in scenarios where hands-free device control is advantageous. As the demand for more natural and accessible human-computer interaction grows, researchers have developed various methods to enable gesture-based control, including wearable devices, external sensors, and voice recognition technologies.

Despite the advancements in this field, existing gesture recognition systems face several limitations that hinder their widespread adoption and effectiveness. Camera-based setups, while common, often struggle with accuracy due to the two-dimensional nature of image capture. This limitation makes it challenging to interpret complex user movements accurately, especially in dynamic environments. Furthermore, in multi-device settings, determining the user's intended target device remains a significant challenge, often resulting in erroneous command execution.

Wearable sensor-based systems, incorporating technologies such as gyroscopes and accelerometers, offer improved gesture detection accuracy. However, these solutions introduce new usability issues, as they require users to wear additional hardware. This requirement can lead to discomfort, particularly for individuals with mobility issues or physical disabilities, thereby limiting the system's practicality in certain environments.

Alternative approaches, such as those utilizing infrared sensors or radio-frequency identification (RFID), have also been explored. While these technologies offer unique advantages, they come with their own set of limitations. Infrared-based systems suffer from a restricted range of operation and can be easily obstructed by environmental factors. RFID systems, while effective at detecting user proximity, often necessitate users to carry RFID tags, compromising the goal of truly hands-free and seamless interaction.

Given the limitations of conventional gesture-based control systems, there is a pressing need for innovative solutions that address these challenges. An ideal system would offer accurate gesture recognition, operate effectively in multi-device environments, and provide a truly hands-free and comfortable user experience. Such advancements would significantly enhance the accessibility and usability of electronic devices, particularly for individuals with mobility challenges, and pave the way for more intuitive human-computer interaction in various settings.

According to the World Health Organization, 15\% of the world’s population, or 1
billion people, experience some form of disability. For these individuals, the ability
to independently control their living environment can be a significant challenge, often
requiring assistance or the use of specialized devices. Conventional home automation
solutions, which rely on physical interfaces like switches and buttons, can be difficult
for those with limited mobility or motor skills to operate. By addressing the needs of
this substantial user group, we hope to contribute to the development of more inclusive
and empowering home automation technologies, ultimately enhancing the quality of
life.

This paper aims to address these challenges by proposing a novel approach to gesture-based control of electronic devices in room environments. Our research focuses on developing a system that overcomes the limitations of existing technologies while providing a more intuitive, accurate, and user-friendly experience for controlling multiple devices through gestures.
% You must have at least 2 lines in the paragraph with the drop letter
% (should never be an issue)

% needed in second column of first page if using \IEEEpubid
%\IEEEpubidadjcol


